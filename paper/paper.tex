\documentclass[twocolumn]{article}
\usepackage{times}
\usepackage{parskip}
\begin{document}
\title{Cyclone: Replication Middleware for NVM Clusters}
\author{Amitabha Roy et al. \\ amitabha.roy@intel.com}
\maketitle
\begin{abstract}
Memory technology in the datacenter is due to undergo a paradigm shift with the
increased use of directly addressable non-volatile memory both in the form of
battery backed non-volatile DIMMs as well as newer memory types such as
3D XPoint. This will allow a broad range of new applications that exploit the
combination of durability with performance far superior to traditional durable
storage media such as NAND flash. However, durability does not guaratee
availability and without a means to survive machine or storage medium failures
these applications cannot match the survivability offered by sophisticated
distributed file systems, databases and key value stores.

Cyclone is replication middleware aimed at helping application developers
targeting new memory types to easily add fault tolerance to their
applications. Cyclone eschews external co-ordination services like zookeeper and
instead replicates RPC calls from clients being servered out of durable memory
across a quorum of clusters using the RAFT consensus protocol. Cyclone also
provides a client side library to enable transparent failover. Finally, Cyclone
is designed to provide at most once semantics to RPC calls. We demonstrate in
this paper that this is sufficient to design applications that support
distributed transactions without any need for the developer to be concerned with
fault recovery - a challenging aspect of designing systems that support
distributed transaction and at the same time are fault tolerant.

Cyclone is also designed to provide maximum performance without being dependent
on high end networking stacks - such as RDMA/Infiniband combinations.  To enable
the lowest possible latency we use DPDK - a userspace networking stack and
designed our implementation of RAFT to use zero copy for the replication
process. To enable maximum throughput under load we extensively batch
replication operations and support multiple concurrent instances of RAFT to
maximize throughput. As a result, Cyclone can replicate as many as 4 million 100
byte RPC messages every second on a very modest commodity 10 GigE network. Under
low load, the latency seen by a client for an RPC call the requires replication
is of the order of 20 us.
\end{abstract}  

\section{Introduction}
TBD
\section{Background: Single Node Programming Model}
Cyclone builds on top of Intel's Non Volatile Memory Library (NVML) and provides
transparent replication to programs written to use NVML for persistence on a
single node. We describe NVML's programming model in this section to provide
adequate background for the rest of this paper.

NVML provides special functions to allocate and free persistent memory as well
as fat pointers to persistent memory to ensure that referential integrity in the
heap is maintained across restarts - since the virtual memory ranges into which
persistent memory is mapped may change.

\begin{figure}
{\scriptsize
\begin{verbatim}

struct ll_node {
  int value;
  TOID(ll_node) next;
};

void insert_after(TOID(ll_node) prev, 
                  TOID(ll_node) new_node)
{
  D_RW(new_node)->next = D_RO(prev)->next;
  D_RW(prev)->next = new_node;
}

\end{verbatim}
}
\caption{A Persistent Linked List of Integers}
\label{fig:example}
\end{figure}

Figure~\ref{fig:example} shows the example of a linked list node and a function
for inserting a new node after a given one. The {\tt TOID} declaration wraps the
linked list node type to produce a fat pointer to it. A {\tt TOID(ll\_node)}
type therefore is a pointer to a linked list node stored in persistent memory.
The {\tt insert\_after} function uses the {\tt D\_RO} and {\tt D\_RW} calls to
convert fat pointers into directly accessible read-only and read-write pointers
respectively. Other than the mechanics of using fat pointers the code entirely
resembles that which a programmer might write for a standard linked list in
DRAM.

\subsection{Failure Atomicity}
Programmers typically find it hard to reason about failures when writing
code. This is a problem when directly manipulating data structures in persistent
memory. Depending on the specifics of the system architecture, updates to
persistent memory can be persisted out of order thereby leaving the data
structure in an inconsistent state. To illustrate this consider a failure during
execution of the code to add a node to a linked list in
Figure~\ref{fig:example}. If either of the two updates to the next fields of the
previous and new node fail to be persisted, the linked list is left in a
disconnected state with one or more connected nodes inaccessible.

NVML provides tools for programmers to write code that is failure
atomic by allowing a block of code to be delimited as a transaction. A
transaction is either completely executed in terms of its effects on
persistent memory or not executed at all. Under the hood, NVML tracks all
updates to persistent memory in a persistent undo log - the log entries are
persisted before the update to the actual memory is allowed. On a failure, the
undo log is used to roll back incomplete transactions. A transaction is
considered complete when it completes execution and the updated objects are
flushed (automatically by NVML) to persistent memory. The log entries for the
transaction are then immediately deleted. The only burden placed on a programmer
for failure atomicity is to delimit transactions and that they must inform NVML
about objects that are updated. Figure~\ref{fig:example_fatomic} shows how to
convert the function for adding a node to a linked list into a failure atomic
one. The only additions are the transaction delimiters and the {\tt TX\_ADD}
calls to mark the objects being updated \emph{before} the actual updates to
them.

\begin{figure}
{ \scriptsize
\begin{verbatim}

struct ll_node {
  int value;
  TOID(ll_node) next;
};

void insert_after(TOID(ll_node) prev, TOID(ll_node) new_node)
{
  TX_BEGIN {
    TX_ADD(new_node);
    D_RW(new_node)->next = D_RO(prev)->next;
    TX_ADD(prev);
    D_RW(prev)->next = new_node;
  } TX_END
}

\end{verbatim}
}
\caption{Failure Atomic Peristent Linked List}
\label{fig:example_fatomic}
\end{figure}

A crucial feature of NVML is that it abstracts the details of the persistence
mechanism away from programmers. This allows the same code to be deployed on
across a range of persistence mechanisms - mapping pages from secondary
storage such as SSDs, using battery backed NVDIMMs, using an onboard battery to
save DRAM state to SSD on failure [FARM] or using directly accessed persistent
memory [PMFS]. Each of these options has different requirements. For example the
NVM used in PMFS requires explicit cache flush (CLFLUSH) and PCOMMIT
instructions. NVDIMMs require only a CLFLUSH to flush the
cacheline and ensure that the line enters the asynchronous data refresh
domain. Using secondary storage is also an option with faster 3DXPoint storage
behind an NVMe interface - this would require an {\tt msync} and OS pagecache
intervention to persist data to secondary storage. NVML seamlessly uses the
right persistence mechanism without requiring code using the library to be aware
of these details.

\section{Replication}
Cyclone is designed to add fault tolerance to NVML client server applications
via replication - a quorum of machines maintains \emph{equivalent} state. Server
state is queried and manipulated via RPC calls from clients. Cyclone's approach
to replication is simple - we replicate the RPC call itself across replicas
rather than replicating every access made during execution of the RPC call,
which would be prohibitively expensive.

Cyclone provides strongly consistent replication - every machines in the quorum
of replicas maintains a log of RPC calls. All RPC calls (including read-only
calls that do not change state and therefore do not need replication) are always
executed by the leader replica. Cyclone treats the RPC call itself as a variable
sized opaque blob of data, leaving the application free to choose how it wishes
to marshall call related details and argument. All machines agree on the
committed sequence of RPC calls in the log. We achieve this by running an
instance of the RAFT [raft] consensus protocol to keep the log of RPC calls on
different machines in synchronization. The standard replicated state machine
approach requires that commands (entries in the log) be committed before they
can be applied to the state machine i.e. in our case - the RPC call be
executed. However, the fact that RPC calls that modify state must themselves be
executed in a crash consistent transaction allows us to overlap execution with
replication (described in the next section).

\subsection{Decoupled Replication}
\label{sec:decouple}
Cyclone decouples execution of the RPC call on any replica from its replication
by performing them in parallel to save time. Normally, this is a problem with
any consenus protocol including RAFT as replication might failt for a number of
reasons primarily due to a change of leader and subsequent of rollback of
previously appended (but not committed) log entries. In order to achieve
decoupled replication therefore Cyclone depends on the ability of NVML to roll
back crash consistent transactions. The ability to roll back a crash consistent
transaction is important to tolerate power failure as incomplete transactions
must be rolled back after recovery from a power failure to preserve the
integrity of the heap. NVML also allows the user to roll back a transaction if
needed. Cyclone exploits this feature by delaying the commit of the failure
atomic transaction after execution till replication is complete. In the event
that replication fails, the transaction is aborted and any changes to the non
volatile heap are rolled back.

A pseudocode description of how the runtime handles decoupled replication is
shown in Figure~\ref{fig:async_rep}.

\begin{figure}
{ \scriptsize
\begin{verbatim}
Initiate replication of RPC call and arguments
TX_BEGIN {
  Execute RPC call and modify local NVM
  Block till replication result known
  if(replication failed) {
    TX_ABORT;
  }
} TX_END
\end{verbatim}
}
\caption{Asynchronous Replication}
\label{fig:async_rep}
\end{figure}

Asynchronous replication provides good performance and we believe would be the
common case for RPC calls replicated via Cyclone. However the program might on
occasion require successful replication to be a precondition to exection. A case
where we find this necessary is output commit - where the RPC call has external
side effects, such as communicating with other machines in the cluster and
therefore crash consistency is insufficient to roll back the execution of the
RPC call in the event that replication fails. We therefore provide users the
ability to declare at the client side that an RPC call should execute
synchronously at the server side. This results in Cyclone ensuring that the RPC
call is successfully replicated before it attempts execution on any of the
replicas.

\subsection{Network Stack}
Replication protocols are expensive to operate - fact that has been underlined in
multiple pieces of research [FARM, in a box]. It would therefore appear that
replication would become the bottleneck in Cyclone. We observed however that
previous work did not implement replication around a high performance network
stack possibly explaining why they chose both high performance hardware (40
GigE+ NICS, custom FPGA designs) as well as specific high performance but
restrictive software APIs (one sided RDMA with metadata co-location, application
compiled down to the FPGA). As a counterpoint, we designed Cyclone around DPDK
[dpdk] a high performance user-space network stack. DPDK provides convenient
access to packets arriving at an ethernet NIC by moving them into packet buffers
with the contained cache lines also directly placed into the last level cache of
the CPU during DMA.

Cyclone separates the available cores on the machine into cores dedicated to
work related to the consensus protocol and cores decidated to running
application code. The division is configurable but Fig [] shows the division for
our testbed. Each RAFT core runs an \emph{independent} instance of the RAFT
protocol (Section~\ref{sec:ind}). RAFT cores pass work to the application cores
through lock-free double ended queues. A RAFT core dispatches an incoming RPC
call for execution simultaneously with sending it out for replication. As
explained in Section~\ref{sec:decouple} this is beneficial for reducing latency
to completion. 

Each RAFT core has a dedicated NIC queue pair. The input queue receives RPC
calls from clients as well as RAFT protocol related traffic. The output queue
only carries RAFT protocol related traffic. Each application core has a
dedicated \emph{output} queue on the NIC. When completed, the response to the
RPC call is sent from the application core via its dedicated output queue. This
division of NIC queues eliminates any need for synchronization between RAFT and
application cores when accessing the NIC.

\subsection{Zero Copy Replication}
Consensus protcols such as RAFT usually have a simple common case path. Once a
quorum leader is elected, the leader's primary job is to transmit new entries
appended to the log to its followers. The primary job of the followers is to
receive these entries and send acknowledgments to the leader. An entry is
considered committed when the leader receives an acknowledgment for an entry
from a majority of the quorum. Followers receive intimation of committed entries
as piggybacked information on subsequent replication messages.

Our aim when designing Cyclone was to be relatively insensitive to the size of
the RPC call - i.e. the size of the RAFT log entry being replicated. Since
Cyclone is targeted towards general purpose applications we intended to support
a range of RPC call sizes without a degradation in performance caused by copying
the RPC call contents for transmission to each follower replica. Hence, we built
a zero copy mechanism for common case of a leader replica sending the same log
entry to a set of follower replicas. To achieve our zero copy goal we explot the
fact that DPDKs packet buffers allow prepending a data without physically moving
the data in the packet buffer. Fig [] shows how the layout of a packet buffer
changes at the leader during replication. The received packets contains only the
RPC call from the client. We prepend a RAFT related header (containing the
previous log index, previous log term, leader's term and leader's commit
point) to packet buffer itself. To this RAFT header we then prepend an IPv4
header. Although we use the raw packet interface from DPDK and do not use IP
based routing, we set IP addresses in a dummy header to use 5-tuple steering in
order to place the packet in the right queue at the receiving NIC. This
composite packet buffer is then sent to every replica. In order to properly
route the packet through switches, we allocate a small buffer containing an
ethernet header for each follower replica and \emph{chain} it in front of the
prepared packet - ensuring we do not need to change the contents for every
replica.

\subsection{Batching}
Although the common case path in RAFT is straightforward, it still adds many
hundreds of cycles for each log entry to be processed. This adds up to a few
microseconds, a significant amount in the face of sub 10 us latency between
nodes with DPDK. We amortize this overhead when operating under load by batching
multiple RPC calls together and treating the batch as a single RAFT log entry.
DPDK by itself can efficiently batch incoming packets under load - allowing the
user to easily receive a burst (currently limited to 32) packets rather than a
single one for a polling call. DPDK further allows chaining packet buffers
together and treating the chain as a single packet. This is convenient for our
zero copy approach and we chain packet mbufs together to form a single packet
as shown in Fig []. Since the entire chain is sent as a single packet when being
replicated, the length of a chain is limited by the size of the coalesced
packet, that must fit into an ethernet jumbo frame of approximately 9K bytes.

\subsection{Multiple RAFT Instances}
\label{sec:ind}
We scale replication performance in Cyclone by using multiple cores on the same
machine to each run an independent instance of the RAFT protocol. For a quorum
of homogeneous machines, we dedicate the same core and NIC queue pair on each
machine to the same instance of RAFT. Each instance maintains an
\emph{independent} log. All the RAFT instances on a machine share the common set
of application cores, dispatching work to them via the (shared) lock-free double
ended queues (implemented as circular buffers in DPDK). It is important to note
that there is no fixed ordering between RPC calls dispatched for execution from
different RAFT instances. We discuss in Sections
~\ref{sec:exec},\ref{sec:example} how a Cyclone user can still impose
determinism for linearizable data structures, while continuing to scale across
the multiple RAFT instances.

Other than the lock free queues for dispatching work to application cores, the
multiple RAFT instances share nothing, being assigned dedicated NIC queue pairs
and hence also dedicated packet buffer pools associated with those queue
pairs. This removes any synchronization bottlenecks on memory allocation. We
note that DPDK uses reference counting to decide when to free a packet
buffer. RAFT cores continue to hold a reference to a packet containing an RPC
call even after dispatching it for execution until it has completed replicating
the contents of the packet and is ready to compact away that entry from its RAFT
log.

\subsection{RPC Call Execution}
\label{sec:exec}
We now describe how RPC calls are executed and results returned to the
client. There are two important aspects of RPC call execution in Cyclone -
steering, and at most once semantics.

An RPC call via Cyclone must traverse a specific RAFT instance (RAFT log) and
specific execution core. These details are provided by the user when making the
RPC call to the client side library provided by Cyclone. The user therefore is
aware of the number of execution cores and RAFT instances and given complete
control of \emph{RPC call steering}.

Cyclone provides at most once semantics. It supports a large but fixed (compile
time configured) number of clients. Each RPC call from a client bears a sequence
number. \emph{Each execution core independently enforces execution of RPC calls
  in increasing sequence numbers}. This is done by storing the last executed
sequence number in a per-core durable datastructure as part of the crash
consistent transaction wrapping the execution of the RPC call itself. Even in
the presence of fail-overs, this durable memory of sequence numbers seen from a
client ensures at most once semantics. In addition to guarding agains repeated
execution of the same call, Cyclone also remembers the \emph{results} of the
last executed RPC call. In order to assist clients (particularly stateless
clients), Cyclone allows the client to query the last seen sequence number and
result for that RPC call from the quorum of servers.


\section{Putting it together - A Durable Replicated Hash Table}
\label{sec:example}
We now demonstrate how Cyclone can be used in practise by putting together an
example of a linearizable hash table - a useful and common data structure
[memcached]. The baseline is a durable hash table written in NVML that runs on a
single machine in client-server mode. The server side maintains the durable
open-addressed hash table and services client RPC calls to insert and delete
key-value pairs and lookup keys. Such a hash table is simple to derive from a
volatile hash table and readily available in NVML [nvml.io] as an example [src
  code link]. To use that (non client-server) example with Cyclone we add a
simple RPC call format, manually marshall and demarshall the (at most two)
arguments to the RPC call in an RPC call handler we wrote (a few tens of lines
of code) and wrote a simple client driver program (a few tens of additional
lines of code).

In order to use the hash table with Cyclone, we simply need to to declare the RPC
call handler as the point of entry to Cyclone at the server end - shown in
Figure []. The client end issues RPC calls via Cyclone's client side library as
shown in Figure []. Crucially neither side involves fault awareness or fault
recovery code - a key goal for Cyclone.

There are two challenges in scaling the performance and improving the utility of
the linearizable and durable hash table. The first is to ensure linearizability
while scaling performance by adding RAFT cores and application cores. The second
is to support distributed transactions - that manipulate a set of key-value
pairs atomically in the face of failure.

\subsection{Scaling}
Scaling the performance of the hash table by using multiple RAFT cores and
execution cores presents determinism related challenges. Two RPC calls being
steered through different RAFT cores can end up executing in different orders on
different replicas even if steered to the same execution core. At the same time,
two RPC calls steered through the same RAFT core but dispatched to different
execution cores can end up completing in different orders in different replicas.

Our solution to this problem in Cyclone is one that applies to most durable
data structures - not just the hash table we consider here. We observed that
replicas of the durable data structure require \emph{per-key determinism}
and not determinism across \emph{all} operations. We therefore steer
operations to the same key to the same RAFT core and the same execution core, by
hashing the key to determine the RAFT core and execution core. The actual
configuration of the hash table therefore can be different across different
replicas - for example due to the insertion of two keys mapping to the same hash
bucket completing in different orders across different replicas. However, since the
results of insert, delete and lookup operations for a key depend only on
previous operations to the same key the replicas provide identical results for
committed operations.

The RPC call steering mechanism described above leaves the programmer free to
use any synchronization mechanism they wish between concurrent accesses from the
application cores to the data structure, since the ordering between accesses
from different cores is allowed to be non-deterministic. For our open addressed
hash table we chose the simple scheme of a per-bucket lock.

\subsection{Distributed Transactions}
A useful primitive for many distributed systems is the ability to run
distributed transactions - an atomic sequence of conditional updates spanning
multiple machines. Distributed transactions have two distinct facets:
atomicity and fault tolerance. Distributed transactions appear atomic with
respect to other atomic transactions and simple updates or reads by virtue of
standard concurrency control techniques. The most common technique is two phase
locking: cquire locks, verify that previously read objects are at the desired
values, update the locked objects and finally release the locks. A survey of
various distributed systems and informal discussions with programmers led us to
the conclusion that this aspect of distributed transactions was fairly easy to
understand and implement. The hard part for most programmers is the second
aspect: fault tolerance. A distributed transaction spans many machines and has a
large footprint to clean up in the event of failure. Failure handling is made
more complex when considering cascading failures - further failures while
recovering from a failure. Cyclone was designed to allow primitives such as
distributed transctions to be added without the programmer needing to worry
about faults or write fault recovery code - as we demonstrate with the hash
table.

We support distributed transactions in our hash table by allowing the user to
specify a group of insert/delete/lookup operations that must occur
atomically. We use the existing bucket locks to implement a simple two phase
locking protocol to implement the transction. A client makes an RPC call to the
server specifying the transaction and the server executes it on the client's
behalf. Figure [] shows pseudocode for the transaction. Execution of the
transaction has four distinct phases: acquire locks, verify values, apply
updates and finally release locks.

We guard against fault tolerance by executing the RPC call on a quorum of
machines via Cyclone. Since execution of the RPC call leaves \emph{externally
  visible side effects} by virtue of the RPC calls it makes we require output
commit [] - the property that the execution is guaranteed to complete once it
starts. This is simply achieved by a flag in the RPC call that turns off
decoupled execution (Section~\ref{sec:decouple}). The flag causes execution to
start only after the RPC call has been successfully replicated.

Next, we need to ensure that although every machine in the quorum makes all
necessary RPC calls, the call itself is executed only once. This is achieved by
using Cyclone's at most once semantics. All machines in the quorum executing the
transaction use the same client identifier and the same montonically increasing
sequence of RPC call numbers. Finally, we need to deal with the fact that
Cyclone only remembers the last executed RPC call for a client. This means that
if different machines in the quorum move at different speeds, some machines will
not receive a response to the RPC call - they will instead receive an error that
tells them that they are too far behind. To ensure that the transaction is still
guaranteed to complete, we use a \emph{different} client identifier for each
phase. Thus, until the transaction completes and a new one is started the
response for any RPC call continues to be available. Figure [] details in
pseudocode the execution of a transaction with these added details. Crucially
there is no need for awareness of faults nor any need to write fault recovery
code. Finally, we need to ensure that the client can reliably query the result
of the transaction. If the leader of the quorum executing the transaction falls
behind it cannot supply the result. We therefore immediately return and
acknowledment to the client as soon as the RPC call for the transaction has been
replicated. We store the result of the transaction in a special data structure
updated through an additional RPC call made at the end of the transaction. The
client queries this data structure to determine the fate of the transaction.

\subsection{Checkpoints}

\section{Evaluation}

\section{Related Work}

\end{document}




\documentclass[twocolumn]{article}
\usepackage{times}

\begin{document}
\title{Cyclone: Fault Tolerance Middleware for NVM Clusters}
\author{Amitabha Roy et al. \\ amitabha.roy@intel.com}
\maketitle
\begin{abstract}
Memory technology in the datacenter is due to undergo a paradigm shift with the
introduction of directly addressable non-volatile memory both in the form of
battery backed non-volatile DIMMs as well as newer memory types such as
3D XPoint. Datacenter applications typically use storage devices to survive
faults that take one more machines offline by combining the persistence offered by
durable storage with availability through replication. Building such fault
tolerant applications is hard and users are forced to either build
them from scratch or modify their applications to fit the interfaces of solutions
such as zookeeper. In addition to a loss of flexibility such applications today
are unable to take advantage of the speed of directly attached non-volatile
memory as opposed to older slower storage in the form of disks and SSDs.

This paper presents Cyclone - fault tolerance middleware that exposes a
replicated and durable heap. This heap can be directly manipulated using loads and
stores. Cyclone transparently provides both failure atomicity as well as
strongly consistent replication thereby giving strong guarantees to programmers
with minimal effort on their part. Under the hood Cyclone decouples modification
of local persistent memory from replication to maximize performance. In
addition, unlike traditional systems that provide strong replication, Cyclone
does not take checkpoints since NVM is directly manipulated by programmers and
therefore up to date till the last executed transaction. Instead, Cyclone
provides a novel NVM copy mechanism to bring up new nodes added to the cluster.
\end{abstract}  

\section{Introduction}

\section{Programming Model}
Cyclone provides users with a replicated 


\section{Failure Atomicity}

\section{Replication}

\section{Evauation}

\subsection{Baseline}

\subsection{Configuration Management}

\subsection{Transaction Processing}

\subsection{Network Function Virtualization}

\end{document}



